% Options for packages loaded elsewhere
\PassOptionsToPackage{unicode}{hyperref}
\PassOptionsToPackage{hyphens}{url}
\PassOptionsToPackage{dvipsnames,svgnames,x11names}{xcolor}
%
\documentclass[
  letterpaper,
  DIV=11,
  numbers=noendperiod]{scrartcl}

\usepackage{amsmath,amssymb}
\usepackage{iftex}
\ifPDFTeX
  \usepackage[T1]{fontenc}
  \usepackage[utf8]{inputenc}
  \usepackage{textcomp} % provide euro and other symbols
\else % if luatex or xetex
  \usepackage{unicode-math}
  \defaultfontfeatures{Scale=MatchLowercase}
  \defaultfontfeatures[\rmfamily]{Ligatures=TeX,Scale=1}
\fi
\usepackage{lmodern}
\ifPDFTeX\else  
    % xetex/luatex font selection
\fi
% Use upquote if available, for straight quotes in verbatim environments
\IfFileExists{upquote.sty}{\usepackage{upquote}}{}
\IfFileExists{microtype.sty}{% use microtype if available
  \usepackage[]{microtype}
  \UseMicrotypeSet[protrusion]{basicmath} % disable protrusion for tt fonts
}{}
\makeatletter
\@ifundefined{KOMAClassName}{% if non-KOMA class
  \IfFileExists{parskip.sty}{%
    \usepackage{parskip}
  }{% else
    \setlength{\parindent}{0pt}
    \setlength{\parskip}{6pt plus 2pt minus 1pt}}
}{% if KOMA class
  \KOMAoptions{parskip=half}}
\makeatother
\usepackage{xcolor}
\setlength{\emergencystretch}{3em} % prevent overfull lines
\setcounter{secnumdepth}{-\maxdimen} % remove section numbering
% Make \paragraph and \subparagraph free-standing
\makeatletter
\ifx\paragraph\undefined\else
  \let\oldparagraph\paragraph
  \renewcommand{\paragraph}{
    \@ifstar
      \xxxParagraphStar
      \xxxParagraphNoStar
  }
  \newcommand{\xxxParagraphStar}[1]{\oldparagraph*{#1}\mbox{}}
  \newcommand{\xxxParagraphNoStar}[1]{\oldparagraph{#1}\mbox{}}
\fi
\ifx\subparagraph\undefined\else
  \let\oldsubparagraph\subparagraph
  \renewcommand{\subparagraph}{
    \@ifstar
      \xxxSubParagraphStar
      \xxxSubParagraphNoStar
  }
  \newcommand{\xxxSubParagraphStar}[1]{\oldsubparagraph*{#1}\mbox{}}
  \newcommand{\xxxSubParagraphNoStar}[1]{\oldsubparagraph{#1}\mbox{}}
\fi
\makeatother


\providecommand{\tightlist}{%
  \setlength{\itemsep}{0pt}\setlength{\parskip}{0pt}}\usepackage{longtable,booktabs,array}
\usepackage{calc} % for calculating minipage widths
% Correct order of tables after \paragraph or \subparagraph
\usepackage{etoolbox}
\makeatletter
\patchcmd\longtable{\par}{\if@noskipsec\mbox{}\fi\par}{}{}
\makeatother
% Allow footnotes in longtable head/foot
\IfFileExists{footnotehyper.sty}{\usepackage{footnotehyper}}{\usepackage{footnote}}
\makesavenoteenv{longtable}
\usepackage{graphicx}
\makeatletter
\def\maxwidth{\ifdim\Gin@nat@width>\linewidth\linewidth\else\Gin@nat@width\fi}
\def\maxheight{\ifdim\Gin@nat@height>\textheight\textheight\else\Gin@nat@height\fi}
\makeatother
% Scale images if necessary, so that they will not overflow the page
% margins by default, and it is still possible to overwrite the defaults
% using explicit options in \includegraphics[width, height, ...]{}
\setkeys{Gin}{width=\maxwidth,height=\maxheight,keepaspectratio}
% Set default figure placement to htbp
\makeatletter
\def\fps@figure{htbp}
\makeatother

\usepackage{booktabs}
\usepackage{longtable}
\usepackage{array}
\usepackage{multirow}
\usepackage{wrapfig}
\usepackage{float}
\usepackage{colortbl}
\usepackage{pdflscape}
\usepackage{tabu}
\usepackage{threeparttable}
\usepackage{threeparttablex}
\usepackage[normalem]{ulem}
\usepackage{makecell}
\usepackage{xcolor}
\KOMAoption{captions}{tableheading}
\makeatletter
\@ifpackageloaded{caption}{}{\usepackage{caption}}
\AtBeginDocument{%
\ifdefined\contentsname
  \renewcommand*\contentsname{Table of contents}
\else
  \newcommand\contentsname{Table of contents}
\fi
\ifdefined\listfigurename
  \renewcommand*\listfigurename{List of Figures}
\else
  \newcommand\listfigurename{List of Figures}
\fi
\ifdefined\listtablename
  \renewcommand*\listtablename{List of Tables}
\else
  \newcommand\listtablename{List of Tables}
\fi
\ifdefined\figurename
  \renewcommand*\figurename{Figure}
\else
  \newcommand\figurename{Figure}
\fi
\ifdefined\tablename
  \renewcommand*\tablename{Table}
\else
  \newcommand\tablename{Table}
\fi
}
\@ifpackageloaded{float}{}{\usepackage{float}}
\floatstyle{ruled}
\@ifundefined{c@chapter}{\newfloat{codelisting}{h}{lop}}{\newfloat{codelisting}{h}{lop}[chapter]}
\floatname{codelisting}{Listing}
\newcommand*\listoflistings{\listof{codelisting}{List of Listings}}
\makeatother
\makeatletter
\makeatother
\makeatletter
\@ifpackageloaded{caption}{}{\usepackage{caption}}
\@ifpackageloaded{subcaption}{}{\usepackage{subcaption}}
\makeatother

\ifLuaTeX
  \usepackage{selnolig}  % disable illegal ligatures
\fi
\usepackage{bookmark}

\IfFileExists{xurl.sty}{\usepackage{xurl}}{} % add URL line breaks if available
\urlstyle{same} % disable monospaced font for URLs
\hypersetup{
  pdftitle={GES 678: Week 2},
  colorlinks=true,
  linkcolor={blue},
  filecolor={Maroon},
  citecolor={Blue},
  urlcolor={Blue},
  pdfcreator={LaTeX via pandoc}}


\title{GES 678: Week 2}
\usepackage{etoolbox}
\makeatletter
\providecommand{\subtitle}[1]{% add subtitle to \maketitle
  \apptocmd{\@title}{\par {\large #1 \par}}{}{}
}
\makeatother
\subtitle{Roles and Responsibilities of a Project Manager}
\author{}
\date{2025-09-03}

\begin{document}
\maketitle

\renewcommand*\contentsname{Table of contents}
{
\hypersetup{linkcolor=}
\setcounter{tocdepth}{3}
\tableofcontents
}

\newpage{}

\subsection{Lecture}\label{lecture}

\subsubsection{Governance}\label{governance}

Within every organization, there is \emph{someone} who helps make sure
everything works together.

Governance is a formal structure for organizations to ensure that IT
investments support business objectives.

It is the process of managing and controlling key IT capability
decisions to improve IT management, ensure compliance, and increase
value from technology investments.

\subsubsection{Components of Governance}\label{components-of-governance}

\begin{itemize}
\tightlist
\item
  Strategic alignment

  \begin{itemize}
  \tightlist
  \item
    Are we all doing the same thing for the same reason?
  \end{itemize}
\item
  Value delivery

  \begin{itemize}
  \tightlist
  \item
    Are we doing IT for IT's sake? What value do we deliver?
  \end{itemize}
\item
  Performance management

  \begin{itemize}
  \tightlist
  \item
    Are we meeting our goals? How is this measured?
  \end{itemize}
\item
  Risk management

  \begin{itemize}
  \tightlist
  \item
    How do we mitigate what goes wrong when it goes wrong? How can we
    identify risks ahead of time?
  \end{itemize}
\end{itemize}

\subsubsection{Governance Structures}\label{governance-structures}

\begin{itemize}
\tightlist
\item
  Steering Committee

  \begin{itemize}
  \tightlist
  \item
    Strategic focus (higher-level)
  \item
    Senior leadership/management
  \item
    Holds some direct funding capabilities.
  \item
    Direct, set, and enforce policies, priorities, and goals.
  \item
    Coordinates external support, such as grant application and citizen
    buy-in.
  \end{itemize}
\item
  Advisory Group/Committee

  \begin{itemize}
  \tightlist
  \item
    Peers/contemporaries/stakeholders
  \item
    Not decision making; more of a recommending group.
  \item
    Identify goals, set objectives, gather information, and bolster
    internal support.
  \item
    Lower-level (e.g.~we need to revamp \emph{this} process, we need to
    improve our web presence).
  \end{itemize}
\item
  External Groups

  \begin{itemize}
  \tightlist
  \item
    Governing agency/organization
  \item
    Professional organizations
  \item
    Standards Committees
  \item
    Citizen Groups
  \item
    Other stakeholders
  \end{itemize}
\item
  Project Management Office (PMO)

  \begin{itemize}
  \tightlist
  \item
    Coordinates among and between projects, including resources, people,
    and schedules.
  \item
    Often seen in larger organizations, where multiple projects occur
    simultaneously.
  \item
    Reduces duplication of efforts across organization.
  \end{itemize}
\end{itemize}

\subsubsection{Exercises}\label{exercises}

\begin{description}
\item[Scenario 1]
You start as a new GIS manager, whose team is working with deprecated
software, old on-prem servers, and older techonology mapping software.
No updates have been budgeted or planned. What are the roles of:
\end{description}

\begin{itemize}
\tightlist
\item
  Steering Committee
\item
  Advisory Committee
\item
  External Groups?
\end{itemize}

\begin{description}
\item[A:]
The steering committee needs to know how and why the upgrades will
advance the organization's mission. The advisory committee will provide
\emph{advice} and recommendations to identify specific objectives and
more technical information related to different parts of the upgrade.
External groups will provide feedback, whether solicited or unsolicited,
which the steering committee can use to make their final decision on
which systems to upgrade (read: fund) when.
\end{description}

\textbf{Q:} Does the steering committee pick the participants of the
advisory committee?

\textbf{A:} It depends on what the organization is and does. The GIS
manager can recommend certain people be on the advisory committee, or
the steering committee can ask the GIS manager to pick, or a mixture of
both.

\begin{description}
\item[Scenario 2]
The GIS team is planning its next web map application and needs to
select among 3 requests. What would be the role of the three groups?
\item[A:]
The steering committee may not have as much of a role because this is a
lower-level question. It may be smaller, more detached, or depend more
specifically on the recommendation of the advisory committee. The
advisory committee would serve more of a peer review role; checking each
others' work and doing research with one another before passing a final
recommendation to the steering committee. The steering committee will
also ensure that the procurement policy is being followed by the
advisory committees and the vendors.
\end{description}

\begin{quote}
\textbf{Note that governance is all still part of Tomlinson's Step 1;
consideration of the strategic purpose.}
\end{quote}

\subsubsection{Roles and
Responsibilities}\label{roles-and-responsibilities}

Each project member should have an assigned role, which roles should be
assigned up front. Likewise, each task and activity should have a
defined role. These roles may change during a project, but they always
need to be clearly defined to the team.

\subsubsection{RACI}\label{raci}

RACI is a framework which defines the activities and decision-making
authorities.

\begin{description}
\item[Responsible]
refers to the completion of a specfic task or activity. This could be
documentation, making a decision, or product development or deployment.
\item[Accountable]
refers to the responsibility of the product owner to ensure
accountability. They must address issues, deal with the impacts of
decisions and results, and focus on the end results of the task.
\item[Consulted]
refers to someone who advises on tasks and activities. People consulted
often have opinions or knowledge that benefit the project and are often
on the advisory committee.
\item[Informed]
refers to keeping the team up to date on tasks and progress. This
includes updates on status, completion of milestones, and schedule
adherence. It is one-way communication, such as press releases and
announcements.
\end{description}

\newpage{}

\textbf{RACI Chart Example}

\begin{longtable}[t]{llll}

\caption{\label{tbl-raci-example}}

\tabularnewline

\toprule
Role & Define Scope & Data Collection & Application Development\\
\midrule
Executive & A & I & I\\
GIS Manager & R & A/C & A/I\\
Data Manager & I/C & R/C & C/A\\
Tech Architect & C/I & C/R & C/I\\
Developer & I/C & I/C & R\\
\addlinespace
Business Analyst & C/I & I/C & I/C\\
\bottomrule

\end{longtable}

\begin{description}
\item[RACI Chart Exercise]
The GIS team has been assigned the project by the City Manager to work
with Public Relations office to create an adopt a hydrant site. The
public can register online to ``adopt'' a fire hydrant by keeping it
clear of snow and debris and report damage. The hydrant data is owned
and maintained by the Fire Department. IT manages the City website
technology, but PR manages web content. Develop a RACI chart for the
project, include all stakeholders and their roles.
\end{description}

\begin{longtable}[t]{l>{\raggedright\arraybackslash}p{2cm}>{\raggedright\arraybackslash}p{2cm}>{\raggedright\arraybackslash}p{2cm}>{\raggedright\arraybackslash}p{2cm}l}

\caption{\label{tbl-raci-exercise}}

\tabularnewline

\toprule
Role & Define Scope & Develop Site & Collect Data & Maintain Data & Maintain Assets\\
\midrule
City Manager & A & I & I & I & I\\
GIS Manager & R & A/C & I/C & A & I\\
GIS Team & I & R & I/C & I & I\\
Public Relations & I & R & I & I & I\\
Public & I & I & I & R & R\\
\addlinespace
Fire Department & I/C & I/C & R/A & R/A & R/A\\
IT Department & I & R & I & I & I\\
\bottomrule

\end{longtable}




\end{document}
